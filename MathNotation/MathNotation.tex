%% Based on a TeXnicCenter-Template by Gyorgy SZEIDL.
%%%%%%%%%%%%%%%%%%%%%%%%%%%%%%%%%%%%%%%%%%%%%%%%%%%%%%%%%%%%%

%------------------------------------------------------------
%
\documentclass[10pt,fleqn]{article}%
%Options -- Point size:  10pt (default), 11pt, 12pt
%        -- Paper size:  letterpaper (default), a4paper, a5paper, b5paper
%                        legalpaper, executivepaper
%        -- Orientation  (portrait is the default)
%                        landscape
%        -- Print size:  oneside (default), twoside
%        -- Quality      final(default), draft
%        -- Title page   notitlepage, titlepage(default)
%        -- Columns      onecolumn(default), twocolumn
%        -- Equation numbering (equation numbers on the right is the default)
%                        leqno
%        -- Displayed equations (centered is the default)
%                        fleqn (equations start at the same distance from the right side)
%        -- Open bibliography style (closed is the default)
%                        openbib
% For instance the command
%           \documentclass[a4paper,12pt,leqno]{article}
% ensures that the paper size is a4, the fonts are typeset at the size 12p
% and the equation numbers are on the left side
%
\usepackage[left=1.0in, right=1.0in, top=0.5in, bottom=1.0in]{geometry}
\usepackage{layout}
\usepackage{parskip}
\usepackage{enumerate}
\usepackage{appendix}
\usepackage{float}
\usepackage{amsmath}
\usepackage{algorithm, algorithmic}
\usepackage{amsfonts}%
\usepackage{amssymb}%
\usepackage{graphicx}
%-------------------------------------------
\newtheorem{theorem}{Theorem}
\newtheorem{acknowledgement}[theorem]{Acknowledgement}
%\newtheorem{algorithm}[theorem]{Algorithm}
\newtheorem{axiom}[theorem]{Axiom}
\newtheorem{case}[theorem]{Case}
\newtheorem{claim}[theorem]{Claim}
\newtheorem{conclusion}[theorem]{Conclusion}
\newtheorem{condition}[theorem]{Condition}
\newtheorem{conjecture}[theorem]{Conjecture}
\newtheorem{corollary}[theorem]{Corollary}
\newtheorem{criterion}[theorem]{Criterion}
\newtheorem{definition}[theorem]{Definition}
\newtheorem{example}[theorem]{Example}
\newtheorem{exercise}[theorem]{Exercise}
\newtheorem{lemma}[theorem]{Lemma}
\newtheorem{notation}[theorem]{Notation}
\newtheorem{problem}[theorem]{Problem}
\newtheorem{proposition}[theorem]{Proposition}
\newtheorem{remark}[theorem]{Remark}
\newtheorem{solution}[theorem]{Solution}
\newtheorem{summary}[theorem]{Summary}
\newenvironment{proof}[1][Proof]{\textbf{#1.} }{\ \rule{0.5em}{0.5em}}

\begin{document}

\title{Math\textbackslash Model Notation\\
for Prediction of Chess Endgame}
\author{Anderson, Callahan, Gutshall}
\date{May 13, 2012}
\maketitle

\begin{abstract}
A quick definition of terms used for our final project.  This will help when we write our final report and will make sure we can communicate easily. 
\end{abstract}

\section{Chess Board Positions}
Let the following notation describe the space of a Chess board,
\begin{equation}
\label{eq:StandardPositions}
\begin{aligned}
	\text{File} &\in \left[a,b,c,d,e,f,g,h\right]\\
	&\in \left[1,2,3,4,5,6,7,8\right]\\
	\text{Rank} &\in \left[1,2,3,4,5,6,7,8\right]\\
\end{aligned}
\end{equation}

\section{Game Pieces}
Let the three game pieces be defined as $Piece_i$, where $i = \left[1,2,3\right]$ represents the piece index,
\begin{equation}
\label{eq:pieces}
\begin{aligned}
	Piece &\in \left[W_k, W_r, B_k\right]\\
	&W_k = \text{White King}\\
	&W_r = \text{White Rook}\\
	&B_k = \text{Black King}\\
\end{aligned}
\end{equation}

\section{Class Labels}
Class labels are defined as the remaining moves till checkmate of $B_k$.  Note, checkmate of $B_k$ is called on the $m^{\text{th}}$ move of $B_k$.
\begin{equation*}
\begin{aligned}
	Class &\in \left[draw,0,1,2,3,4, \dots, m\right]\\
	&\in \left[-1,0,1,2,3,4, \dots, m\right]
\end{aligned}
\end{equation*}

\section{Instance of a Game}
Let a instance of a game be defined as $Game_k$ where $k = \left[1,2,3,\dots\right]$ is the game index, 
%
\begin{equation*}
\begin{aligned}
	Game_k &= \left[Piece_i \left\{file, rank\right\}, Class\right]\\
	&= \left[\left\{2, 1\right\},\left\{7,9\right\},\left\{6,3\right\},8\right] \\
	&= \left[W_k \left\{2, 1\right\}, W_r \left\{7, 9\right\}, B_k \left\{6, 3\right\}, 8\right]\\
	&= \left[W_k \left\{b, 1\right\}, W_r \left\{g, 9\right\}, B_k \left\{f, 3\right\}, 8\right]\\
\end{aligned}
\end{equation*}

\section{Parameters}
Several parameter functions are used to classify or achieve an objective function, these parameter functions are labeled as $\mathbf{\Phi}\left(\mathbf{x}\right)_j$.  Where, $j = \left\{1,2,3,4,\dots\right\}$ is a index for the parameter function and $\mathbf{x}$ could be any parameter used in the function.

\end{document}
